% Options for packages loaded elsewhere
\PassOptionsToPackage{unicode}{hyperref}
\PassOptionsToPackage{hyphens}{url}
\documentclass[
]{article}
\usepackage{xcolor}
\usepackage{amsmath,amssymb}
\setcounter{secnumdepth}{-\maxdimen} % remove section numbering
\usepackage{iftex}
\ifPDFTeX
  \usepackage[T1]{fontenc}
  \usepackage[utf8]{inputenc}
  \usepackage{textcomp} % provide euro and other symbols
\else % if luatex or xetex
  \usepackage{unicode-math} % this also loads fontspec
  \defaultfontfeatures{Scale=MatchLowercase}
  \defaultfontfeatures[\rmfamily]{Ligatures=TeX,Scale=1}
\fi
\usepackage{lmodern}
\ifPDFTeX\else
  % xetex/luatex font selection
\fi
% Use upquote if available, for straight quotes in verbatim environments
\IfFileExists{upquote.sty}{\usepackage{upquote}}{}
\IfFileExists{microtype.sty}{% use microtype if available
  \usepackage[]{microtype}
  \UseMicrotypeSet[protrusion]{basicmath} % disable protrusion for tt fonts
}{}
\makeatletter
\@ifundefined{KOMAClassName}{% if non-KOMA class
  \IfFileExists{parskip.sty}{%
    \usepackage{parskip}
  }{% else
    \setlength{\parindent}{0pt}
    \setlength{\parskip}{6pt plus 2pt minus 1pt}}
}{% if KOMA class
  \KOMAoptions{parskip=half}}
\makeatother
\setlength{\emergencystretch}{3em} % prevent overfull lines
\providecommand{\tightlist}{%
  \setlength{\itemsep}{0pt}\setlength{\parskip}{0pt}}
\usepackage{bookmark}
\IfFileExists{xurl.sty}{\usepackage{xurl}}{} % add URL line breaks if available
\urlstyle{same}
\hypersetup{
  hidelinks,
  pdfcreator={LaTeX via pandoc}}

\author{}
\date{}

\begin{document}

{
\setcounter{tocdepth}{3}
\tableofcontents
}
\section{Tolio: Plataforma Web de Economía Colaborativa para Préstamo de
Herramientas y
Servicios}\label{tolio-plataforma-web-de-economuxeda-colaborativa-para-pruxe9stamo-de-herramientas-y-servicios}

\textbf{Proyecto Final de Licenciatura en Ciencias de la Computación}\\
\textbf{Autor:} Tobias Gatti (LU: 136467)\\
\textbf{Universidad Nacional del Sur}\\
\textbf{Departamento de Ciencias e Ingeniería de la Computación}\\
\textbf{Año:} 2024

\begin{center}\rule{0.5\linewidth}{0.5pt}\end{center}

\subsection{Índice}\label{uxedndice}

\begin{enumerate}
\def\labelenumi{\arabic{enumi}.}
\tightlist
\item
  \hyperref[1-introducciuxf3n-al-marketplace]{Introducción al
  Marketplace}
\item
  \hyperref[2-mecanismos-de-pagos]{Mecanismos de Pagos}
\item
  \hyperref[3-mi-propuesta-tolio]{Mi Propuesta: Tolio}
\item
  \hyperref[4-implementaciuxf3n-y-consideraciones-tuxe9cnicas]{Implementación
  y Consideraciones Técnicas}
\item
  \hyperref[5-conclusiones-y-trabajo-futuro]{Conclusiones y Trabajo
  Futuro}
\end{enumerate}

\begin{center}\rule{0.5\linewidth}{0.5pt}\end{center}

\subsection{1. Introducción al
Marketplace}\label{introducciuxf3n-al-marketplace}

\subsubsection{1.1 Definición y
Concepto}\label{definiciuxf3n-y-concepto}

Un \textbf{marketplace} (mercado digital) es una plataforma en línea que
actúa como intermediario entre múltiples vendedores o prestadores de
servicios y compradores o usuarios. A diferencia de una tienda en línea
tradicional donde una única empresa vende sus propios productos, un
marketplace permite que diversos proveedores independientes ofrezcan sus
bienes o servicios en un espacio común, beneficiándose de la
infraestructura tecnológica, el tráfico y la confianza que proporciona
la plataforma.

El marketplace digital representa la evolución natural del concepto
milenario de ``mercado'' físico, donde comerciantes se congregaban en un
espacio común para ofrecer sus productos. La transformación digital de
este concepto ha democratizado el comercio, permitiendo que cualquier
persona con un producto o servicio pueda acceder a una audiencia global
sin necesidad de invertir en infraestructura propia.

\subsubsection{1.2 Historia y Evolución}\label{historia-y-evoluciuxf3n}

La historia del marketplace digital moderno comenzó a mediados de la
década de 1990 y principios de los 2000, marcando una revolución en la
forma en que las personas compran y venden bienes y servicios.

\paragraph{Pioneros del Marketplace
Digital}\label{pioneros-del-marketplace-digital}

\textbf{eBay (1995)}\\
Fundado por Pierre Omidyar, eBay fue uno de los primeros marketplaces en
línea y revolucionó el concepto de subastas en internet. Con más de 25
años de existencia, eBay permitió que particulares vendieran
directamente a otros particulares (modelo C2C - Consumer-to-Consumer),
creando un ecosistema de comercio electrónico descentralizado. En 2002,
la adquisición de PayPal fortaleció significativamente su capacidad de
procesamiento de pagos, consolidando su posición como líder del mercado.

\textbf{Amazon (2000-2006)}\\
Aunque Amazon comenzó como una librería en línea en 1994, su
transformación en marketplace ocurrió alrededor del año 2000. Jeff Bezos
tuvo la visión de trasladar el modelo de un centro comercial físico al
mundo digital, permitiendo que terceros vendieran sus productos en la
plataforma de Amazon. Este cambio estratégico permitió que Amazon pasara
de ser un minorista a convertirse en una plataforma que integra
catálogos de múltiples vendedores, ofreciendo una variedad prácticamente
ilimitada de productos.

\textbf{Mercado Libre (1999)}\\
Fundado en Argentina por Marcos Galperín, Mercado Libre se convirtió en
el marketplace líder de América Latina. Comenzando en 1999, la
plataforma se expandió rápidamente por toda la región, adaptando el
modelo de marketplace a las particularidades del mercado
latinoamericano, incluyendo métodos de pago locales y logística adaptada
a las necesidades regionales.

\textbf{Alibaba (1999)}\\
Jack Ma fundó Alibaba observando el éxito de Amazon y eBay, pero
enfocándose inicialmente en el comercio B2B (Business-to-Business).
Alibaba conectó fabricantes chinos con compradores internacionales,
democratizando el acceso al comercio internacional y convirtiéndose en
el marketplace más grande del mundo en términos de volumen de
transacciones.

\subsubsection{1.3 Modelos Funcionales de
Marketplace}\label{modelos-funcionales-de-marketplace}

Los marketplaces pueden clasificarse según diferentes criterios, cada
uno con características y dinámicas particulares:

\paragraph{Por Tipo de Transacción}\label{por-tipo-de-transacciuxf3n}

\textbf{B2C (Business-to-Consumer)}\\
Empresas o profesionales venden productos o servicios directamente a
consumidores finales. Ejemplos: Amazon, Asos, Uber. Este modelo es ideal
para negocios establecidos que buscan ampliar su alcance sin invertir en
infraestructura propia de e-commerce.

\textbf{B2B (Business-to-Business)}\\
Transacciones entre empresas. Alibaba.com es el ejemplo más prominente,
conectando fabricantes con distribuidores y empresas que necesitan
insumos o productos al por mayor.

\textbf{C2C (Consumer-to-Consumer)}\\
Permite a particulares vender productos a otros particulares. Facebook
Marketplace, eBay y Mercado Libre (en su modalidad de venta entre
particulares) son ejemplos claros de este modelo, que democratiza el
comercio y permite monetizar bienes usados o artesanales.

\textbf{Modelos Mixtos}\\
Plataformas como Amazon Prime combinan elementos B2C y C2C, permitiendo
que tanto empresas establecidas como vendedores individuales ofrezcan
productos en la misma plataforma.

\paragraph{Por Modelo de
Monetización}\label{por-modelo-de-monetizaciuxf3n}

\textbf{Modelo de Comisiones}\\
El más popular en la actualidad. El marketplace recibe un porcentaje de
cada transacción realizada en la plataforma. Este modelo alinea los
incentivos de la plataforma con el éxito de los vendedores, ya que ambos
se benefician del volumen de ventas.

\textbf{Tarifas por Anuncios}\\
Los vendedores pagan por publicitar sus productos o servicios, ya sea
por clic (CPC) o por impresiones (CPM). Este modelo es común en
marketplaces con alto tráfico donde la visibilidad es valiosa.

\textbf{Suscripciones}\\
Los vendedores o compradores pagan una tarifa recurrente (mensual o
anual) por acceder a la plataforma o a funciones premium. Amazon Seller
Central utiliza este modelo para vendedores profesionales.

\textbf{Tarifas de Listado}\\
Los vendedores pagan por publicar sus productos en la plataforma,
independientemente de si se venden o no. Este modelo es menos común en
marketplaces modernos debido a que puede desincentivar la participación.

\textbf{Modelo Freemium}\\
Ofrece acceso básico gratuito con funciones limitadas, cobrando por
características avanzadas como mejor posicionamiento, análisis
detallados o herramientas de marketing.

\paragraph{Por Modelo Operativo}\label{por-modelo-operativo}

\textbf{Marketplace Administrado}\\
El intermediario verifica vendedores, gestiona procesos logísticos y de
mantenimiento, garantizando calidad y confianza. Amazon FBA (Fulfillment
by Amazon) es un ejemplo donde Amazon maneja el almacenamiento y envío.

\textbf{Marketplace On Demand}\\
Ofrecen servicios bajo demanda inmediata del consumidor. Uber, Rappi y
Glovo son ejemplos donde el servicio se solicita y se presta en tiempo
real.

\textbf{Marketplace de Nicho}\\
Se especializan en un sector o grupo demográfico específico. Etsy
(artesanías), Airbnb (alojamiento) y Upwork (freelancers) son ejemplos
de marketplaces que dominan nichos específicos.

\subsubsection{1.4 Características Clave de un Marketplace
Exitoso}\label{caracteruxedsticas-clave-de-un-marketplace-exitoso}

Para que un marketplace funcione efectivamente, debe incorporar ciertos
elementos fundamentales:

\begin{enumerate}
\def\labelenumi{\arabic{enumi}.}
\item
  \textbf{Efecto de Red}: El valor de la plataforma aumenta
  exponencialmente con cada nuevo usuario (tanto vendedores como
  compradores).
\item
  \textbf{Confianza y Seguridad}: Sistemas de verificación, reseñas,
  calificaciones y garantías que protejan a ambas partes.
\item
  \textbf{Facilidad de Uso}: Interfaz intuitiva que permita a usuarios
  con diferentes niveles de habilidad técnica participar fácilmente.
\item
  \textbf{Procesamiento de Pagos Seguro}: Integración con pasarelas de
  pago confiables que protejan la información financiera.
\item
  \textbf{Sistema de Reputación}: Mecanismos que permitan construir
  confianza a través de historial de transacciones y valoraciones.
\item
  \textbf{Resolución de Disputas}: Procesos claros para manejar
  conflictos entre compradores y vendedores.
\end{enumerate}

\begin{center}\rule{0.5\linewidth}{0.5pt}\end{center}

\subsection{2. Mecanismos de Pagos}\label{mecanismos-de-pagos}

\subsubsection{2.1 Panorama de Pagos en Argentina
2024}\label{panorama-de-pagos-en-argentina-2024}

Argentina ha experimentado una transformación digital acelerada en su
ecosistema de pagos durante los últimos años. En 2024, el país se
caracteriza por una fuerte tendencia hacia la digitalización y la
consolidación de las billeteras virtuales como protagonistas principales
del sistema de pagos.

\paragraph{Estadísticas Clave}\label{estaduxedsticas-clave}

\begin{itemize}
\tightlist
\item
  \textbf{75\%} de los argentinos mayores de 18 años posee al menos una
  billetera digital, igualando la tenencia de tarjetas de débito.
\item
  \textbf{73\%} de los adultos utiliza dos o más billeteras digitales.
\item
  Los pagos con código QR crecieron un \textbf{53\%} en 2024.
\item
  Las transferencias inmediatas aumentaron un \textbf{43\%} interanual.
\item
  Las tarjetas prepago experimentaron un crecimiento del \textbf{97.3\%}
  en cantidad de operaciones.
\end{itemize}

\subsubsection{2.2 Métodos de Pago
Dominantes}\label{muxe9todos-de-pago-dominantes}

\paragraph{Billeteras Digitales
(E-wallets)}\label{billeteras-digitales-e-wallets}

Las billeteras digitales son el motor de la transformación digital de
los pagos en Argentina:

\textbf{Mercado Pago}\\
La plataforma dominante en Argentina, con la mayor variedad de métodos
de pago y fácil integración con plataformas de e-commerce. Ofrece: -
Pagos con tarjetas de crédito y débito - Transferencias bancarias -
Pagos en cuotas sin interés - Códigos QR interoperables - Links de pago
- Integración con múltiples plataformas (WooCommerce, Tienda Nube,
Shopify)

\textbf{MODO}\\
Impulsada por más de 30 bancos argentinos, MODO permite la
interoperabilidad entre múltiples entidades bancarias y plataformas. Es
la segunda billetera más usada (31.6\% de la población) y se destaca
por: - Transferencias gratuitas entre usuarios - Pagos QR en comercios -
Respaldo del sistema bancario tradicional

\textbf{Ualá}\\
Con su tarjeta prepaga Mastercard y aplicación móvil, Ualá se enfoca en
la inclusión financiera. Es la tercera billetera más utilizada (20.6\%)
y ofrece: - Tarjeta prepaga sin costos de mantenimiento - Inversiones en
fondos comunes - Préstamos personales - Solución para negocios (Ualá
Bis)

\textbf{Cuenta DNI y BNA+}\\
Billeteras respaldadas por bancos estatales (Banco Provincia y Banco
Nación) que buscan promover la inclusión financiera de sectores no
bancarizados.

\paragraph{Pagos con Código QR}\label{pagos-con-cuxf3digo-qr}

Los pagos QR han revolucionado la experiencia de pago en Argentina. El
Banco Central (BCRA) ha impulsado la interoperabilidad de los pagos QR,
lo que significa que todos los lectores QR deben aceptar pagos de todas
las tarjetas y billeteras digitales. Esto ha eliminado la fragmentación
del mercado y facilitado la adopción masiva.

\paragraph{Transferencias 3.0 (Pagos
A2A)}\label{transferencias-3.0-pagos-a2a}

El sistema de Transferencias 3.0 ha impulsado fuertemente los pagos de
cuenta a cuenta (Account-to-Account), facilitando transacciones entre
comercios y usuarios mediante códigos QR. El 73.8\% de las
transferencias inmediatas utilizan CVU (Clave Virtual Uniforme), el
identificador de las billeteras digitales.

\paragraph{Tarjetas de Crédito y
Débito}\label{tarjetas-de-cruxe9dito-y-duxe9bito}

Aunque las billeteras digitales están ganando terreno, las tarjetas
siguen siendo fundamentales:

\textbf{En e-commerce:} - Tarjetas de crédito: 35-39\% de las
transacciones - Tarjetas de débito: 13.5-18\% de las transacciones - La
preferencia por el pago en cuotas sin interés es un factor clave para el
uso de tarjetas de crédito en Argentina

\textbf{En puntos de venta físicos:} - Efectivo: 27\% - Tarjetas de
crédito: 25\% - Tarjetas de débito: 25\% - Billeteras digitales: 18\%

\subsubsection{2.3 Pasarelas de Pago Disponibles en
Argentina}\label{pasarelas-de-pago-disponibles-en-argentina}

Las pasarelas de pago son intermediarios esenciales que permiten a los
negocios recibir pagos de manera segura y eficiente en línea.

\paragraph{Principales Pasarelas}\label{principales-pasarelas}

\textbf{Mercado Pago}\\
- \textbf{Ventajas}: Mayor adopción, múltiples opciones de integración,
cuotas, links de pago, QR - \textbf{Desventajas}: Comisiones elevadas
(pueden llegar al 6-8\%), posibles retenciones fiscales - \textbf{Ideal
para}: Pequeños y medianos negocios, marketplaces

\textbf{Todo Pago (Payway SPS)}\\
- \textbf{Ventajas}: Respaldada por Prisma/Visa, seguridad antifraude
avanzada, promociones bancarias - \textbf{Desventajas}: Proceso de
integración más complejo - \textbf{Ideal para}: Negocios establecidos
que buscan seguridad y respaldo bancario

\textbf{Payway (Decidir)}\\
- \textbf{Ventajas}: Gateway de alto nivel, costos bajos por
transacción, flexibilidad para desarrollos a medida -
\textbf{Desventajas}: Requiere mayor conocimiento técnico -
\textbf{Ideal para}: Empresas grandes con equipos de desarrollo

\textbf{Rebill}\\
- \textbf{Ventajas}: Integración de diversas formas de pago (tarjetas,
transferencias, efectivo, billeteras), pagos en cuotas, Transferencias
3.0 - \textbf{Desventajas}: Menor reconocimiento de marca que Mercado
Pago - \textbf{Ideal para}: Negocios que buscan flexibilidad en métodos
de pago

\textbf{PayU}\\
- \textbf{Ventajas}: Presencia internacional, acepta tarjetas y
transferencias - \textbf{Desventajas}: Comisiones variables según el
método de pago - \textbf{Ideal para}: Negocios con operaciones
internacionales

\subsubsection{2.4 Consideraciones Legales y
Regulatorias}\label{consideraciones-legales-y-regulatorias}

El Banco Central de la República Argentina (BCRA) regula el sistema de
pagos y mantiene un registro de Proveedores de Servicios de Pago (PSP),
incluyendo adquirentes y proveedores de cuentas de pago. Las principales
regulaciones incluyen:

\begin{itemize}
\tightlist
\item
  \textbf{Interoperabilidad obligatoria} de códigos QR
\item
  \textbf{Límites de comisiones} para ciertos tipos de transacciones
\item
  \textbf{Requisitos de seguridad} para el procesamiento de pagos
\item
  \textbf{Protección al consumidor} en transacciones digitales
\item
  \textbf{Declaración obligatoria} de ingresos por alquileres ante AFIP
\end{itemize}

\begin{center}\rule{0.5\linewidth}{0.5pt}\end{center}

\subsection{3. Mi Propuesta: Tolio}\label{mi-propuesta-tolio}

\subsubsection{3.1 Identificación del
Problema}\label{identificaciuxf3n-del-problema}

En la actualidad, existe una paradoja en el consumo de herramientas y
equipos: muchas personas necesitan herramientas específicas para
proyectos ocasionales, pero la compra de estas herramientas representa
una inversión económica significativa para un uso esporádico.
Simultáneamente, millones de herramientas permanecen inactivas en
garajes y depósitos, representando capital inmovilizado que podría
generar valor.

\paragraph{Problemas Específicos
Identificados:}\label{problemas-especuxedficos-identificados}

\begin{enumerate}
\def\labelenumi{\arabic{enumi}.}
\item
  \textbf{Barrera Económica}: Herramientas profesionales (taladros,
  amoladoras, equipos de jardinería) pueden costar entre \$50,000 y
  \$500,000 ARS, un gasto difícil de justificar para un uso ocasional.
\item
  \textbf{Desperdicio de Recursos}: Se estima que las herramientas
  domésticas se utilizan menos del 5\% de su vida útil, representando un
  desperdicio masivo de recursos.
\item
  \textbf{Falta de Acceso a Servicios Profesionales}: Encontrar
  profesionales confiables para changas (plomería, electricidad,
  carpintería) es difícil, especialmente sin referencias personales.
\item
  \textbf{Desconfianza en Transacciones P2P}: Las plataformas existentes
  no ofrecen suficientes garantías de seguridad, verificación de
  identidad o protección en caso de daños.
\item
  \textbf{Impacto Ambiental}: La producción excesiva de herramientas que
  se usan mínimamente contribuye significativamente a la huella de
  carbono global.
\end{enumerate}

\subsubsection{3.2 Solución Propuesta:
Tolio}\label{soluciuxf3n-propuesta-tolio}

Tolio es una plataforma web de economía colaborativa que conecta a
propietarios de herramientas y prestadores de servicios con usuarios que
necesitan acceder temporalmente a estos recursos. La plataforma facilita
todo el proceso desde la publicación, búsqueda y reserva, hasta la
gestión de transacciones y reputación.

\paragraph{Propuesta de Valor}\label{propuesta-de-valor}

\textbf{Para Propietarios/Prestadores:} - Monetizar herramientas y
equipos inactivos - Generar ingresos pasivos sin esfuerzo adicional -
Contribuir a la economía circular y sostenibilidad - Control total sobre
disponibilidad y precios - Protección mediante sistema de verificación
de usuarios

\textbf{Para Usuarios/Solicitantes:} - Acceso económico a herramientas
sin necesidad de comprarlas - Búsqueda geolocalizada de herramientas
cercanas - Acceso a profesionales verificados para servicios - Sistema
de reseñas para tomar decisiones informadas - Comunicación directa con
propietarios

\textbf{Para la Sociedad:} - Reducción del consumo excesivo y la
producción innecesaria - Optimización del uso de recursos existentes -
Disminución de la huella de carbono - Fortalecimiento de la economía
local y colaborativa - Generación de confianza en transacciones entre
particulares

\subsubsection{3.3 Modelo de Negocio}\label{modelo-de-negocio}

Tolio opera bajo un modelo de marketplace bidireccional, donde tanto la
oferta (propietarios/prestadores) como la demanda
(usuarios/solicitantes) son igualmente importantes para el éxito de la
plataforma.

\paragraph{Estrategia de
Monetización}\label{estrategia-de-monetizaciuxf3n}

Actualmente, Tolio se encuentra en fase de desarrollo y validación de
mercado, por lo que \textbf{no implementa comisiones ni procesamiento de
pagos integrado}. Esta decisión estratégica se basa en varios factores:

\begin{enumerate}
\def\labelenumi{\arabic{enumi}.}
\item
  \textbf{Enfoque en Adopción}: Priorizar el crecimiento de la base de
  usuarios sin fricciones económicas adicionales.
\item
  \textbf{Validación del Modelo}: Confirmar que existe demanda real
  antes de implementar infraestructura de pagos compleja.
\item
  \textbf{Simplicidad Operativa}: Evitar complejidades legales,
  contables y fiscales en la etapa inicial.
\item
  \textbf{Flexibilidad para Usuarios}: Permitir que usuarios y
  propietarios acuerden términos de pago que les resulten más
  convenientes.
\end{enumerate}

\textbf{Modelo de Monetización Futuro:}

Una vez validado el producto-mercado fit, Tolio podría implementar: -
\textbf{Comisión por transacción}: 5-10\% sobre el valor del alquiler -
\textbf{Suscripciones premium}: Funcionalidades avanzadas para usuarios
frecuentes - \textbf{Publicidad destacada}: Posicionamiento premium de
publicaciones - \textbf{Servicios adicionales}: Seguros, verificaciones
premium, logística

\subsubsection{3.4 Diferenciación: ¿Por Qué No Implementar Pagos
Integrados?}\label{diferenciaciuxf3n-por-quuxe9-no-implementar-pagos-integrados}

Esta decisión merece una explicación detallada, ya que es una de las
características distintivas de Tolio frente a competidores
internacionales como Hygglio.

\paragraph{Razones Técnicas y de
Producto}\label{razones-tuxe9cnicas-y-de-producto}

\textbf{1. Complejidad de Implementación}\\
Integrar un sistema de pagos robusto requiere: - Cumplimiento de
normativas PCI-DSS para seguridad de datos de tarjetas - Integración con
múltiples pasarelas de pago - Manejo de diferentes métodos de pago
(tarjetas, transferencias, billeteras) - Sistema de escrow (retención de
fondos) para proteger a ambas partes - Gestión de reembolsos y disputas
- Infraestructura de reconciliación contable

\textbf{2. Consideraciones Legales en Argentina}\\
El alquiler de bienes entre particulares en Argentina presenta desafíos
legales específicos:

\begin{itemize}
\item
  \textbf{Responsabilidad Civil}: Según el Código Civil y Comercial
  argentino, el locador (propietario) es responsable de conservar el
  inmueble o bien y reparar deterioros no imputables al inquilino. Sin
  un seguro de responsabilidad civil adecuado, la plataforma podría
  enfrentar riesgos legales.
\item
  \textbf{Seguros Obligatorios}: Para alquileres formales, se requieren
  seguros de caución o garantías. Implementar esto para cada transacción
  de herramientas sería prohibitivamente complejo y costoso.
\item
  \textbf{Declaración ante AFIP}: Los contratos de alquiler deben ser
  declarados ante la Administración Federal de Ingresos Públicos. Esto
  añade una capa de complejidad administrativa que no es práctica para
  alquileres de corto plazo de herramientas.
\item
  \textbf{Retenciones Fiscales}: Las plataformas que procesan pagos
  deben actuar como agentes de retención, lo que implica obligaciones
  fiscales complejas.
\end{itemize}

\textbf{3. Inspiración en Modelos Exitosos: Hygglio}\\
Hygglio, la plataforma líder de alquiler P2P en los países nórdicos,
procesa pagos internamente y toma una comisión del 20\% (50\% para
drones). Sin embargo, opera en un contexto legal y cultural diferente:

\begin{itemize}
\tightlist
\item
  \textbf{Marco Legal Claro}: Los países nórdicos tienen regulaciones
  claras sobre economía colaborativa
\item
  \textbf{Cultura de Confianza}: Alta confianza social que reduce
  riesgos de fraude
\item
  \textbf{Seguros Integrados}: Hygglio Care ofrece seguros automáticos
  en cada transacción
\item
  \textbf{Verificación con BankID}: Sistema de identificación digital
  gubernamental
\end{itemize}

En Argentina, estas condiciones no se cumplen completamente, lo que hace
más riesgoso procesar pagos sin la infraestructura legal y de seguros
adecuada.

\textbf{4. Enfoque en el Valor Principal}\\
Al no procesar pagos, Tolio puede enfocarse en: - Perfeccionar la
experiencia de búsqueda y descubrimiento - Construir un sistema de
reputación robusto - Desarrollar funcionalidades de verificación de
identidad - Optimizar la comunicación entre usuarios - Mejorar la
geolocalización y búsqueda por proximidad

\textbf{5. Flexibilidad para los Usuarios}\\
Permitir que usuarios acuerden pagos directamente ofrece ventajas: -
Pueden usar el método de pago que prefieran (efectivo, transferencia,
Mercado Pago, etc.) - Evitan comisiones de la plataforma - Mayor
flexibilidad en negociaciones de precio - Posibilidad de acuerdos
informales para vecinos o conocidos

\subsubsection{3.5 Inspiración y
Referencias}\label{inspiraciuxf3n-y-referencias}

Tolio se inspira en plataformas exitosas de economía colaborativa:

\textbf{Hygglio (Países Nórdicos)}\\
- Modelo P2P puro para alquiler de herramientas - Sistema de seguridad
``Hygglio Care'' - Verificación de identidad con BankID - Comisión del
20\% sobre transacciones

\textbf{Airbnb}\\
- Marketplace bidireccional exitoso - Sistema de reseñas bidireccional -
Verificación de identidad robusta - Gestión de confianza en
transacciones de alto valor

\textbf{Mercado Libre}\\
- Adaptación a las particularidades del mercado latinoamericano -
Sistema de reputación basado en calificaciones - Múltiples métodos de
pago locales - Resolución de disputas

\textbf{Wallapop/Facebook Marketplace}\\
- Modelo de transacciones directas entre usuarios - La plataforma
facilita la conexión, no procesa pagos - Enfoque en simplicidad y
adopción masiva

\begin{center}\rule{0.5\linewidth}{0.5pt}\end{center}

\subsection{4. Implementación y Consideraciones
Técnicas}\label{implementaciuxf3n-y-consideraciones-tuxe9cnicas}

\subsubsection{4.1 Arquitectura del
Sistema}\label{arquitectura-del-sistema}

Tolio está construido sobre una arquitectura moderna de aplicación web
full-stack, siguiendo las mejores prácticas de desarrollo de software y
priorizando escalabilidad, seguridad y experiencia de usuario.

\paragraph{Arquitectura de Alto Nivel}\label{arquitectura-de-alto-nivel}

\begin{verbatim}
┌─────────────────────────────────────────────────────────────┐
│                     CAPA DE PRESENTACIÓN                     │
│  Next.js 15 (App Router) + React 19 + TypeScript            │
│  - Páginas Server-Side Rendered (SSR)                        │
│  - Componentes Client-Side para interactividad              │
│  - Internacionalización (next-intl)                          │
└─────────────────────────────────────────────────────────────┘
                              │
                              ▼
┌─────────────────────────────────────────────────────────────┐
│                      CAPA DE LÓGICA                          │
│  API Routes (Next.js) + Server Actions                       │
│  - Autenticación (NextAuth.js)                               │
│  - Validación (Zod)                                          │
│  - Manejo de errores                                         │
└─────────────────────────────────────────────────────────────┘
                              │
                              ▼
┌─────────────────────────────────────────────────────────────┐
│                    CAPA DE PERSISTENCIA                      │
│  Prisma ORM + PostgreSQL                                     │
│  - Modelos de datos relacionales                             │
│  - Migraciones versionadas                                   │
│  - Índices optimizados                                       │
└─────────────────────────────────────────────────────────────┘
                              │
                              ▼
┌─────────────────────────────────────────────────────────────┐
│                   SERVICIOS EXTERNOS                         │
│  - AWS S3 (almacenamiento de imágenes)                       │
│  - Vercel Blob (almacenamiento alternativo)                  │
│  - Resend (emails transaccionales)                           │
│  - Leaflet + OpenStreetMap (mapas)                           │
│  - Face-API.js (verificación facial)                         │
└─────────────────────────────────────────────────────────────┘
\end{verbatim}

\subsubsection{4.2 Stack Tecnológico}\label{stack-tecnoluxf3gico}

\paragraph{Frontend}\label{frontend}

\textbf{Next.js 15 (App Router)}\\
Framework de React que permite renderizado del lado del servidor (SSR),
generación estática (SSG) y rutas API integradas. Beneficios: - SEO
optimizado mediante renderizado del lado del servidor - Carga inicial
rápida con hidratación progresiva - Optimización automática de imágenes
y fuentes - Code splitting automático para mejor performance

\textbf{React 19}\\
Biblioteca de JavaScript para construir interfaces de usuario
interactivas con un modelo de componentes reutilizables.

\textbf{TypeScript}\\
Superset de JavaScript que añade tipado estático, reduciendo errores en
tiempo de desarrollo y mejorando la mantenibilidad del código.

\textbf{Tailwind CSS}\\
Framework de CSS utility-first que permite desarrollo rápido con diseño
consistente y responsive.

\textbf{shadcn/ui}\\
Colección de componentes de UI accesibles y personalizables construidos
con Radix UI y Tailwind CSS.

\textbf{React Hook Form + Zod}\\
Gestión de formularios con validación de esquemas, proporcionando una
experiencia de usuario fluida con validación en tiempo real.

\paragraph{Backend}\label{backend}

\textbf{Next.js API Routes}\\
Endpoints de API serverless integrados en Next.js, permitiendo lógica
del lado del servidor sin necesidad de un servidor separado.

\textbf{Prisma ORM}\\
Object-Relational Mapping moderno que proporciona: - Type-safety
completo en consultas a la base de datos - Migraciones automáticas y
versionadas - Introspección de esquema - Cliente de base de datos
optimizado

\textbf{PostgreSQL}\\
Base de datos relacional robusta y escalable, ideal para datos
estructurados con relaciones complejas.

\textbf{NextAuth.js}\\
Solución completa de autenticación para Next.js que soporta: - Múltiples
proveedores OAuth (Google, GitHub, etc.) - Autenticación con
credenciales - Sesiones seguras - Protección CSRF

\paragraph{Servicios Externos y APIs}\label{servicios-externos-y-apis}

\textbf{AWS S3 / Vercel Blob}\\
Almacenamiento de objetos para imágenes de perfil, fotos de productos y
documentos de verificación.

\textbf{Resend}\\
Servicio de emails transaccionales para notificaciones, confirmaciones
de reserva y comunicaciones del sistema.

\textbf{Leaflet + React-Leaflet}\\
Biblioteca de mapas interactivos de código abierto, integrada con
OpenStreetMap para: - Visualización de ubicaciones de items/servicios -
Búsqueda por proximidad geográfica - Marcadores personalizados - Cálculo
de distancias

\textbf{Face-API.js}\\
Biblioteca de reconocimiento facial basada en TensorFlow.js para
verificación de identidad mediante comparación de selfies con
documentos.

\textbf{NSFWJS}\\
Modelo de machine learning para detectar contenido inapropiado en
imágenes subidas por usuarios.

\subsubsection{4.3 Modelo de Datos}\label{modelo-de-datos}

El esquema de base de datos de Tolio está diseñado para soportar las
funcionalidades principales de la plataforma de manera eficiente y
escalable.

\paragraph{Entidades Principales}\label{entidades-principales}

\textbf{User (Usuario)}\\
Almacena información de usuarios, incluyendo: - Datos personales
(nombre, email, teléfono) - Credenciales de autenticación - Estado de
verificación de identidad - Imagen de perfil y biografía - Conexiones
con servicios de pago (Stripe, MercadoPago) - preparado para futuro -
Relaciones con items, servicios, reservas, mensajes, notificaciones

\textbf{Item (Herramienta/Equipo)}\\
Representa herramientas o equipos disponibles para alquiler: -
Información básica (título, descripción, categoría) - Precio y tipo de
precio (por hora/día) - Depósito de seguridad - Ubicación geográfica
(latitud, longitud) - Características y fotos - Estado de disponibilidad

\textbf{Service (Servicio Profesional)}\\
Representa servicios ofrecidos por profesionales: - Información del
servicio (título, descripción, categoría) - Precio por hora o
personalizado - Badge de profesional matriculado - Área de servicio
geográfica - Disponibilidad y características

\textbf{Booking (Reserva de Item)}\\
Gestiona reservas de herramientas: - Fechas de inicio y fin - Precio
total - Estado (PENDING, CONFIRMED, CANCELLED, COMPLETED) - Relaciones
con item, borrower (inquilino) y owner (propietario)

\textbf{ServiceBooking (Reserva de Servicio)}\\
Gestiona contrataciones de servicios: - Fecha de inicio y fin (opcional)
- Horas estimadas - Precio personalizado si aplica - Estado de la
reserva - Relaciones con servicio, cliente y proveedor

\textbf{Review (Reseña)}\\
Sistema de reputación bidireccional: - Calificación (1-5 estrellas) -
Comentario textual - Respuesta del propietario/prestador - Vinculación
con reserva específica para evitar reseñas falsas

\textbf{Notification (Notificación)}\\
Sistema de notificaciones en tiempo real: - Tipo de notificación
(solicitud de reserva, confirmación, mensaje, etc.) - Contenido y título
- Estado de lectura - Enlaces a recursos relacionados - Metadata
adicional en formato JSON

\textbf{Message (Mensaje)}\\
Sistema de mensajería entre usuarios: - Contenido del mensaje - Estado
de lectura - Relación con reserva (opcional) - Sender y receiver

\textbf{Verification (Verificación de Identidad)}\\
Gestiona el proceso de verificación de identidad: - Tipo de verificación
(DNI, selfie, etc.) - Estado (PENDING, APPROVED, REJECTED) - Datos del
documento - URLs de imágenes (selfie, frente y dorso del documento) -
Scores de coincidencia facial y liveness detection - Metadata del
proceso de verificación

\paragraph{Relaciones Clave}\label{relaciones-clave}

El modelo de datos implementa relaciones complejas para garantizar
integridad referencial:

\begin{itemize}
\tightlist
\item
  \textbf{Un usuario} puede tener múltiples items, servicios, reservas
  (como borrower y como owner), mensajes y notificaciones
\item
  \textbf{Un item} pertenece a un usuario y puede tener múltiples
  reservas y reseñas
\item
  \textbf{Una reserva} está vinculada a un item, un borrower y un owner,
  y puede tener una reseña
\item
  \textbf{Las reseñas} están vinculadas a una reserva específica para
  garantizar que solo usuarios que completaron una transacción puedan
  dejar reseñas
\end{itemize}

\subsubsection{4.4 Funcionalidades
Principales}\label{funcionalidades-principales}

\paragraph{1. Autenticación y Gestión de
Usuarios}\label{autenticaciuxf3n-y-gestiuxf3n-de-usuarios}

\textbf{Registro y Login}\\
- Registro con email y contraseña - Autenticación OAuth con Google y
GitHub - Verificación de email mediante tokens - Recuperación de
contraseña - Sesiones seguras con NextAuth.js

\textbf{Verificación de Identidad}\\
Proceso de verificación en múltiples pasos: 1. Usuario sube selfie en
tiempo real 2. Usuario escanea código PDF417 del DNI argentino 3.
Usuario sube fotos del frente y dorso del DNI 4. Sistema extrae datos
del PDF417 (nombre, apellido, número de documento, fecha de nacimiento)
5. Face-API.js compara la selfie con la foto del DNI 6. Se calcula un
score de coincidencia facial 7. Administrador revisa y aprueba/rechaza
la verificación

Este proceso garantiza que los usuarios son quienes dicen ser,
aumentando la confianza en la plataforma.

\paragraph{2. Publicación de Items y
Servicios}\label{publicaciuxf3n-de-items-y-servicios}

\textbf{Creación de Publicaciones}\\
Formulario multi-paso para publicar items o servicios: - Información
básica (título, descripción, categoría) - Precio y condiciones -
Ubicación (opcional, con mapa interactivo) - Características y
especificaciones - Carga de múltiples imágenes con preview - Validación
de contenido inapropiado con NSFWJS

\textbf{Gestión de Publicaciones}\\
- Edición de publicaciones existentes - Activación/desactivación de
disponibilidad - Eliminación con confirmación - Vista previa de cómo se
verá la publicación

\paragraph{3. Búsqueda y
Descubrimiento}\label{buxfasqueda-y-descubrimiento}

\textbf{Búsqueda Avanzada}\\
- Búsqueda por texto (título, descripción, características) - Filtros
por categoría y subcategoría - Filtro por rango de precio - Filtro por
tipo de precio (hora/día) - Ordenamiento (más reciente, precio,
popularidad)

\textbf{Búsqueda Geolocalizada}\\
- Mapa interactivo con marcadores de items/servicios - Búsqueda por
radio de distancia - Cálculo de distancia desde la ubicación del usuario
- Filtros combinados (categoría + ubicación + precio)

\textbf{Navegación por Categorías}\\
- Categorías principales con iconos visuales - Subcategorías para
refinamiento - Contadores de items disponibles por categoría

\paragraph{4. Sistema de Reservas}\label{sistema-de-reservas}

\textbf{Flujo de Reserva para Items}\\
1. Usuario selecciona fechas de inicio y fin 2. Sistema calcula precio
total basado en días/horas 3. Usuario envía solicitud de reserva 4.
Propietario recibe notificación 5. Propietario acepta o rechaza la
reserva 6. Si se acepta, ambas partes reciben confirmación 7. Al
finalizar el período, propietario marca como completada 8. Ambas partes
pueden dejar reseñas

\textbf{Flujo de Reserva para Servicios}\\
1. Usuario describe el trabajo necesario 2. Selecciona fecha preferida y
horas estimadas 3. Envía solicitud al prestador 4. Prestador puede
proponer precio personalizado 5. Cliente acepta o negocia 6. Se confirma
la reserva 7. Al completarse, se marca como completada 8. Sistema de
reseñas bidireccional

\textbf{Gestión de Disponibilidad}\\
- Calendario visual de disponibilidad - Bloqueo automático de fechas
reservadas - Prevención de reservas superpuestas - Notificaciones de
nuevas solicitudes

\paragraph{5. Sistema de Mensajería}\label{sistema-de-mensajeruxeda}

\textbf{Chat Integrado}\\
- Mensajería en tiempo real entre usuarios - Vinculación de
conversaciones con reservas específicas - Indicadores de mensajes no
leídos - Historial de conversaciones - Notificaciones de nuevos mensajes

\paragraph{6. Sistema de Reseñas y
Reputación}\label{sistema-de-reseuxf1as-y-reputaciuxf3n}

\textbf{Reseñas Verificadas}\\
- Solo usuarios que completaron una transacción pueden dejar reseñas -
Calificación de 1 a 5 estrellas - Comentario textual opcional -
Respuesta del propietario/prestador - Timestamp de creación

\textbf{Cálculo de Reputación}\\
- Promedio de calificaciones recibidas - Número total de reseñas -
Visualización de distribución de calificaciones - Badge de
``verificado'' para usuarios con identidad confirmada

\paragraph{7. Panel de Control
(Dashboard)}\label{panel-de-control-dashboard}

\textbf{Para Propietarios/Prestadores}\\
- Resumen de ganancias (preparado para futuro) - Lista de reservas
pendientes, confirmadas y completadas - Gestión de publicaciones -
Estadísticas de visualizaciones - Calendario de disponibilidad

\textbf{Para Usuarios/Solicitantes}\\
- Historial de reservas - Gastos (preparado para futuro) -
Items/servicios favoritos - Mensajes y notificaciones

\paragraph{8. Sistema de
Notificaciones}\label{sistema-de-notificaciones}

\textbf{Tipos de Notificaciones}\\
- Nueva solicitud de reserva - Reserva confirmada - Reserva cancelada -
Reserva completada - Nuevo mensaje recibido - Nueva reseña recibida -
Recordatorios de devolución

\textbf{Canales de Notificación}\\
- Notificaciones in-app con contador - Emails transaccionales
(preparado) - Push notifications (futuro)

\subsubsection{4.5 Seguridad y Privacidad}\label{seguridad-y-privacidad}

\paragraph{Medidas de Seguridad
Implementadas}\label{medidas-de-seguridad-implementadas}

\textbf{Autenticación y Autorización}\\
- Contraseñas hasheadas con bcrypt (12 rounds) - Tokens JWT seguros para
sesiones - Protección CSRF en formularios - Validación de permisos en
cada endpoint - Rate limiting para prevenir ataques de fuerza bruta

\textbf{Protección de Datos}\\
- Validación de entrada con Zod en todos los formularios - Sanitización
de datos antes de almacenar en base de datos - Encriptación de datos
sensibles - HTTPS obligatorio en producción - Headers de seguridad (CSP,
X-Frame-Options, etc.)

\textbf{Privacidad}\\
- Ubicación exacta opcional (usuarios pueden compartir solo
ciudad/barrio) - Datos de contacto ocultos hasta confirmar reserva -
Opción de eliminar cuenta y datos personales - Cumplimiento con
principios de GDPR (aunque no es obligatorio en Argentina)

\textbf{Verificación de Contenido}\\
- Detección automática de contenido inapropiado en imágenes con NSFWJS -
Moderación de reseñas para detectar lenguaje ofensivo - Sistema de
reportes de usuarios

\subsubsection{4.6 Desafíos Técnicos y
Soluciones}\label{desafuxedos-tuxe9cnicos-y-soluciones}

\paragraph{Desafío 1: Verificación de Identidad
Confiable}\label{desafuxedo-1-verificaciuxf3n-de-identidad-confiable}

\textbf{Problema}: Garantizar que los usuarios son quienes dicen ser sin
requerir procesos manuales costosos.

\textbf{Solución Implementada}: - Integración de Face-API.js para
reconocimiento facial - Lectura de código PDF417 del DNI argentino con
ZXing - Comparación automática de selfie con foto del documento -
Cálculo de score de coincidencia facial - Revisión manual final para
casos ambiguos

\textbf{Resultado}: Sistema de verificación robusto que balancea
automatización con supervisión humana.

\paragraph{Desafío 2: Búsqueda Geolocalizada
Eficiente}\label{desafuxedo-2-buxfasqueda-geolocalizada-eficiente}

\textbf{Problema}: Permitir búsquedas por proximidad geográfica sin
comprometer la privacidad ni el rendimiento.

\textbf{Solución Implementada}: - Almacenamiento de coordenadas
(latitud, longitud) en base de datos - Índices en campos de ubicación
para consultas rápidas - Cálculo de distancia usando fórmula de
Haversine - Integración con Leaflet para visualización en mapa -
Geocodificación con OpenStreetMap Nominatim

\textbf{Resultado}: Búsquedas geográficas rápidas con visualización
intuitiva en mapa.

\paragraph{Desafío 3: Prevención de Reservas
Conflictivas}\label{desafuxedo-3-prevenciuxf3n-de-reservas-conflictivas}

\textbf{Problema}: Evitar que un item sea reservado por múltiples
usuarios en las mismas fechas.

\textbf{Solución Implementada}: - Validación de disponibilidad antes de
crear reserva - Transacciones atómicas en base de datos - Bloqueo
optimista con timestamps - Actualización en tiempo real del calendario
de disponibilidad

\textbf{Resultado}: Sistema robusto que previene conflictos de reservas.

\paragraph{Desafío 4: Escalabilidad de
Imágenes}\label{desafuxedo-4-escalabilidad-de-imuxe1genes}

\textbf{Problema}: Almacenar y servir múltiples imágenes de alta calidad
sin impactar el rendimiento.

\textbf{Solución Implementada}: - Almacenamiento en AWS S3 / Vercel Blob
- Optimización automática de imágenes con Next.js Image - Lazy loading
de imágenes - Compresión antes de subir - CDN para distribución global

\textbf{Resultado}: Carga rápida de imágenes con costos de
almacenamiento optimizados.

\paragraph{Desafío 5:
Internacionalización}\label{desafuxedo-5-internacionalizaciuxf3n}

\textbf{Problema}: Preparar la plataforma para múltiples idiomas y
regiones.

\textbf{Solución Implementada}: - Integración de next-intl para i18n -
Archivos de traducción en JSON - Detección automática de idioma del
navegador - Selector de idioma en UI - Formateo de fechas y monedas
según locale

\textbf{Resultado}: Plataforma preparada para expansión internacional.

\subsubsection{4.7 Decisiones de Diseño
Clave}\label{decisiones-de-diseuxf1o-clave}

\paragraph{Decisión 1: Next.js App Router vs Pages
Router}\label{decisiuxf3n-1-next.js-app-router-vs-pages-router}

\textbf{Elección}: App Router (Next.js 15)

\textbf{Justificación}: - Server Components por defecto para mejor
performance - Layouts anidados para mejor organización - Streaming y
Suspense para carga progresiva - Mejor integración con React 19 - Futuro
de Next.js

\paragraph{Decisión 2: PostgreSQL vs
MongoDB}\label{decisiuxf3n-2-postgresql-vs-mongodb}

\textbf{Elección}: PostgreSQL

\textbf{Justificación}: - Datos altamente relacionales (usuarios, items,
reservas, reseñas) - Necesidad de integridad referencial - Consultas
complejas con JOINs - Transacciones ACID para reservas - Mejor soporte
de Prisma para SQL

\paragraph{Decisión 3: Prisma vs
TypeORM}\label{decisiuxf3n-3-prisma-vs-typeorm}

\textbf{Elección}: Prisma

\textbf{Justificación}: - Type-safety superior - Developer experience
excepcional - Migraciones automáticas - Prisma Studio para debugging -
Mejor documentación y comunidad

\paragraph{Decisión 4: Autenticación Propia vs
NextAuth.js}\label{decisiuxf3n-4-autenticaciuxf3n-propia-vs-nextauth.js}

\textbf{Elección}: NextAuth.js

\textbf{Justificación}: - Solución battle-tested y segura - Soporte para
múltiples proveedores OAuth - Gestión de sesiones integrada - Menor
superficie de ataque - Mantenimiento por la comunidad

\paragraph{Decisión 5: Mapas: Google Maps vs Leaflet +
OpenStreetMap}\label{decisiuxf3n-5-mapas-google-maps-vs-leaflet-openstreetmap}

\textbf{Elección}: Leaflet + OpenStreetMap

\textbf{Justificación}: - Completamente gratuito y open-source - Sin
límites de uso ni costos - Personalización completa - Comunidad activa -
Datos de OpenStreetMap de alta calidad

\begin{center}\rule{0.5\linewidth}{0.5pt}\end{center}

\subsection{5. Conclusiones y Trabajo
Futuro}\label{conclusiones-y-trabajo-futuro}

\subsubsection{5.1 Logros Alcanzados}\label{logros-alcanzados}

El desarrollo de Tolio ha cumplido exitosamente con los objetivos
planteados al inicio del proyecto:

\paragraph{Objetivos Técnicos
Cumplidos}\label{objetivos-tuxe9cnicos-cumplidos}

\begin{enumerate}
\def\labelenumi{\arabic{enumi}.}
\item
  \textbf{Plataforma Web Funcional}: Se desarrolló una aplicación web
  completa y funcional que permite a usuarios publicar, buscar y
  reservar herramientas y servicios.
\item
  \textbf{Sistema de Autenticación Robusto}: Implementación de
  NextAuth.js con soporte para múltiples proveedores OAuth y
  autenticación con credenciales.
\item
  \textbf{Verificación de Identidad}: Sistema de verificación biométrica
  que combina reconocimiento facial, lectura de PDF417 y validación
  manual.
\item
  \textbf{Búsqueda Geolocalizada}: Integración de mapas interactivos con
  búsqueda por proximidad geográfica.
\item
  \textbf{Sistema de Reputación}: Implementación de reseñas verificadas
  vinculadas a transacciones completadas.
\item
  \textbf{Arquitectura Escalable}: Diseño modular y escalable que
  permite crecimiento futuro sin refactorización mayor.
\end{enumerate}

\paragraph{Aprendizajes Técnicos}\label{aprendizajes-tuxe9cnicos}

\textbf{Desarrollo Full-Stack Moderno}\\
El proyecto permitió profundizar en tecnologías de vanguardia como
Next.js 15, React 19, TypeScript y Prisma, adquiriendo experiencia
práctica en desarrollo full-stack moderno.

\textbf{Integración de Servicios Externos}\\
Experiencia integrando múltiples APIs y servicios externos (AWS S3,
Resend, OpenStreetMap, Face-API.js), comprendiendo los desafíos de
dependencias externas.

\textbf{Diseño de Bases de Datos Relacionales}\\
Modelado de datos complejos con múltiples relaciones, índices
optimizados y transacciones atómicas.

\textbf{Seguridad en Aplicaciones Web}\\
Implementación de mejores prácticas de seguridad: autenticación,
autorización, validación de entrada, protección CSRF, hashing de
contraseñas.

\textbf{UX/UI y Accesibilidad}\\
Diseño de interfaces intuitivas y accesibles utilizando shadcn/ui,
Tailwind CSS y principios de diseño responsive.

\subsubsection{5.2 Lecciones Aprendidas}\label{lecciones-aprendidas}

\paragraph{Lección 1: La Importancia de la Validación de
Mercado}\label{lecciuxf3n-1-la-importancia-de-la-validaciuxf3n-de-mercado}

Inicialmente, el proyecto contemplaba implementar procesamiento de pagos
integrado. Sin embargo, la investigación sobre aspectos legales,
fiscales y de seguros en Argentina reveló que esta funcionalidad
añadiría complejidad significativa sin validar primero la demanda del
mercado.

\textbf{Aprendizaje}: Es fundamental validar el producto-mercado fit
antes de invertir en funcionalidades complejas. El enfoque MVP (Minimum
Viable Product) permite iterar rápidamente basándose en feedback real de
usuarios.

\paragraph{Lección 2: El Balance entre Automatización y Supervisión
Humana}\label{lecciuxf3n-2-el-balance-entre-automatizaciuxf3n-y-supervisiuxf3n-humana}

El sistema de verificación de identidad combina automatización
(reconocimiento facial, lectura de PDF417) con revisión manual.
Inicialmente se consideró un sistema completamente automatizado, pero se
identificó que la supervisión humana es necesaria para casos ambiguos.

\textbf{Aprendizaje}: La automatización completa no siempre es la mejor
solución. Un enfoque híbrido puede ofrecer mejor balance entre
eficiencia y precisión.

\paragraph{Lección 3: La Complejidad de la Economía
Colaborativa}\label{lecciuxf3n-3-la-complejidad-de-la-economuxeda-colaborativa}

Desarrollar una plataforma de economía colaborativa implica no solo
desafíos técnicos, sino también consideraciones legales, de confianza y
de diseño de incentivos.

\textbf{Aprendizaje}: Las plataformas de dos lados (two-sided
marketplaces) requieren equilibrar cuidadosamente las necesidades de
ambos grupos de usuarios (oferta y demanda) y construir mecanismos de
confianza robustos.

\paragraph{Lección 4: Open Source como Ventaja
Competitiva}\label{lecciuxf3n-4-open-source-como-ventaja-competitiva}

La decisión de usar tecnologías open-source (Leaflet, OpenStreetMap,
PostgreSQL) en lugar de alternativas comerciales (Google Maps, MongoDB
Atlas) resultó en: - Cero costos de licenciamiento - Mayor control y
personalización - Independencia de proveedores - Comunidad activa para
soporte

\textbf{Aprendizaje}: Las soluciones open-source de calidad pueden
ofrecer ventajas significativas para startups y proyectos con
presupuesto limitado.

\subsubsection{5.3 Trabajo Futuro}\label{trabajo-futuro}

\paragraph{Corto Plazo (3-6 meses)}\label{corto-plazo-3-6-meses}

\textbf{1. Lanzamiento Beta y Validación de Mercado}\\
- Lanzar versión beta con usuarios seleccionados - Recopilar feedback
cualitativo y cuantitativo - Iterar sobre funcionalidades basándose en
uso real - Medir métricas clave: tasa de conversión, retención, NPS

\textbf{2. Optimización de Performance}\\
- Implementar caching con Redis - Optimizar consultas a base de datos -
Implementar lazy loading y code splitting adicional - Monitoreo de
performance con Vercel Analytics

\textbf{3. Mejoras en Búsqueda}\\
- Implementar búsqueda full-text con PostgreSQL o Algolia - Filtros
avanzados (disponibilidad, calificación mínima, distancia) - Búsqueda
por voz - Recomendaciones personalizadas basadas en historial

\textbf{4. Sistema de Notificaciones Mejorado}\\
- Implementar push notifications con service workers - Notificaciones
por email con templates personalizados - Preferencias de notificación
granulares - Resumen diario/semanal de actividad

\paragraph{Mediano Plazo (6-12 meses)}\label{mediano-plazo-6-12-meses}

\textbf{1. Implementación de Pagos Integrados} (si se valida demanda)\\
- Integración con Mercado Pago y/o Stripe - Sistema de escrow para
proteger a ambas partes - Gestión de comisiones de plataforma -
Facturación automática - Cumplimiento fiscal (retenciones, declaraciones
AFIP)

\textbf{2. Sistema de Seguros}\\
- Asociación con aseguradoras para ofrecer seguros opcionales - Seguro
de responsabilidad civil para propietarios - Seguro de daños para items
de alto valor - Proceso de reclamos integrado

\textbf{3. Aplicación Móvil Nativa}\\
- Desarrollo de app iOS y Android con React Native - Notificaciones push
nativas - Acceso a cámara para verificación de identidad -
Geolocalización en tiempo real - Modo offline para consultar reservas

\textbf{4. Gamificación y Engagement}\\
- Sistema de badges y logros - Programa de referidos con incentivos -
Niveles de usuario (bronce, plata, oro, platino) - Descuentos por
fidelidad - Comunidad y foros

\paragraph{Largo Plazo (1-2 años)}\label{largo-plazo-1-2-auxf1os}

\textbf{1. Expansión Geográfica}\\
- Lanzamiento en otras ciudades de Argentina - Expansión a países
vecinos (Uruguay, Chile, Paraguay) - Adaptación a regulaciones locales -
Partnerships con empresas de logística

\textbf{2. Categorías Adicionales}\\
- Alquiler de espacios (estudios de fotografía, salas de ensayo) -
Alquiler de vehículos entre particulares - Equipamiento deportivo y
recreativo - Equipos electrónicos y tecnología

\textbf{3. Marketplace B2B}\\
- Sección para empresas que alquilan equipamiento industrial - Contratos
de largo plazo - Facturación B2B - Gestión de flotas de herramientas

\textbf{4. Inteligencia Artificial y Machine Learning}\\
- Recomendaciones personalizadas con ML - Detección de fraude con
anomaly detection - Pricing dinámico basado en demanda - Chatbot de
atención al cliente con NLP - Moderación automática de contenido

\textbf{5. Sostenibilidad y Impacto Social}\\
- Métricas de impacto ambiental (CO2 ahorrado, productos no fabricados)
- Certificación B Corp - Programa de donación de herramientas a
comunidades - Educación sobre economía circular

\subsubsection{5.4 Impacto Potencial}\label{impacto-potencial}

\paragraph{Impacto Económico}\label{impacto-econuxf3mico}

\textbf{Para Usuarios Individuales}\\
- \textbf{Ahorro estimado}: Un usuario que alquila en lugar de comprar
puede ahorrar entre 60-80\% del costo de adquisición. -
\textbf{Generación de ingresos}: Propietarios pueden generar ingresos
pasivos de \$10,000-\$50,000 ARS mensuales alquilando herramientas
inactivas.

\textbf{Para la Economía Local}\\
- Creación de microemprendimientos - Optimización del uso de capital
existente - Reducción de barreras de entrada para proyectos de mejora
del hogar - Fomento de la economía colaborativa

\paragraph{Impacto Ambiental}\label{impacto-ambiental}

\textbf{Reducción de Consumo}\\
- \textbf{Producción evitada}: Si 1000 usuarios alquilan en lugar de
comprar, se evita la producción de \textasciitilde1000 herramientas. -
\textbf{Emisiones de CO2}: La producción de herramientas eléctricas
genera \textasciitilde50-100 kg CO2 por unidad. Evitar 1000 producciones
= 50-100 toneladas de CO2 ahorradas. - \textbf{Residuos electrónicos}:
Reducción de e-waste al extender la vida útil de herramientas
existentes.

\textbf{Economía Circular}\\
- Promoción de modelos de acceso sobre propiedad - Incentivo para
fabricantes de diseñar productos más duraderos - Conciencia sobre
consumo responsable

\paragraph{Impacto Social}\label{impacto-social}

\textbf{Inclusión Económica}\\
- Acceso a herramientas profesionales para personas de bajos ingresos -
Oportunidades de emprendimiento sin inversión inicial alta -
Democratización del acceso a servicios profesionales

\textbf{Construcción de Comunidad}\\
- Fortalecimiento de lazos vecinales - Confianza en transacciones entre
particulares - Cultura de colaboración sobre competencia

\textbf{Educación y Capacitación}\\
- Aprendizaje de nuevas habilidades al acceder a herramientas
especializadas - Transferencia de conocimiento entre usuarios -
Empoderamiento para proyectos DIY (Do It Yourself)

\subsubsection{5.5 Reflexiones Finales}\label{reflexiones-finales}

El desarrollo de Tolio ha sido un proyecto ambicioso que integra
múltiples áreas de las ciencias de la computación: ingeniería de
software, bases de datos, seguridad informática, interfaces de usuario,
desarrollo web full-stack e integración de sistemas. Más allá de los
desafíos técnicos, el proyecto aborda una problemática real con
potencial de impacto social, económico y ambiental significativo.

La decisión de no implementar procesamiento de pagos en la versión
inicial, aunque pueda parecer contraintuitiva para un marketplace,
refleja un enfoque pragmático que prioriza la validación de mercado, la
simplicidad operativa y el cumplimiento legal. Esta decisión está
inspirada en el principio de ``hacer cosas que no escalan'' en las
etapas tempranas, permitiendo aprender rápidamente de usuarios reales
antes de invertir en infraestructura compleja.

El proyecto Tolio demuestra que es posible crear plataformas
tecnológicas sofisticadas utilizando herramientas open-source y
modernas, sin depender de grandes inversiones iniciales. La arquitectura
escalable y modular permite evolucionar la plataforma basándose en
feedback real y necesidades emergentes del mercado.

Finalmente, Tolio representa una oportunidad de contribuir a la
transición hacia modelos de consumo más sostenibles y colaborativos,
demostrando que la tecnología puede ser un habilitador de cambio social
positivo cuando se diseña con propósito y consideración por las
necesidades reales de las personas.

\begin{center}\rule{0.5\linewidth}{0.5pt}\end{center}

\subsection{Referencias}\label{referencias}

\begin{enumerate}
\def\labelenumi{\arabic{enumi}.}
\tightlist
\item
  \textbf{Marketplaces y Economía Colaborativa}

  \begin{itemize}
  \tightlist
  \item
    Sviokla, J. (1996). ``Marketspace: Creating Value in the Digital
    Age''
  \item
    Hagiu, A., \& Wright, J. (2015). ``Multi-sided platforms''.
    International Journal of Industrial Organization
  \item
    Parker, G., Van Alstyne, M., \& Choudary, S. P. (2016). ``Platform
    Revolution''
  \end{itemize}
\item
  \textbf{Sistemas de Pago en Argentina}

  \begin{itemize}
  \tightlist
  \item
    Banco Central de la República Argentina (BCRA). ``Registro de
    Proveedores de Servicios de Pago''
  \item
    PaymentsCMI. (2024). ``Argentina Payment Methods Report 2024''
  \item
    Rebill. (2024). ``Métodos de Pago en Argentina''
  \end{itemize}
\item
  \textbf{Aspectos Legales}

  \begin{itemize}
  \tightlist
  \item
    Código Civil y Comercial de la Nación Argentina
  \item
    Ley 27.551 - Ley de Alquileres (modificada por Ley 27.737)
  \item
    Argentina.gob.ar. ``Contratos de alquiler: derechos y obligaciones''
  \end{itemize}
\item
  \textbf{Tecnologías Utilizadas}

  \begin{itemize}
  \tightlist
  \item
    Next.js Documentation. https://nextjs.org/docs
  \item
    Prisma Documentation. https://www.prisma.io/docs
  \item
    NextAuth.js Documentation. https://next-auth.js.org
  \item
    Leaflet Documentation. https://leafletjs.com
  \item
    Face-API.js. https://github.com/justadudewhohacks/face-api.js
  \end{itemize}
\item
  \textbf{Casos de Estudio}

  \begin{itemize}
  \tightlist
  \item
    Hygglio. ``The Nordic Sharing Economy Platform''
  \item
    Airbnb. ``Building Trust in a Two-Sided Marketplace''
  \item
    Mercado Libre. ``E-commerce in Latin America''
  \end{itemize}
\end{enumerate}

\begin{center}\rule{0.5\linewidth}{0.5pt}\end{center}

\textbf{Anexos}

\begin{itemize}
\tightlist
\item
  Anexo A: Diagramas de Arquitectura Detallados
\item
  Anexo B: Esquema Completo de Base de Datos
\item
  Anexo C: Capturas de Pantalla de la Plataforma
\item
  Anexo D: Código de Componentes Clave
\item
  Anexo E: Resultados de Testing y Validación
\end{itemize}

\begin{center}\rule{0.5\linewidth}{0.5pt}\end{center}

\textbf{Agradecimientos}

Agradezco al Departamento de Ciencias e Ingeniería de la Computación de
la Universidad Nacional del Sur por la formación académica recibida, a
los profesores que guiaron este proyecto, y a la comunidad open-source
cuyas herramientas hicieron posible este desarrollo.

\begin{center}\rule{0.5\linewidth}{0.5pt}\end{center}

\emph{Documento generado el 7 de Diciembre de 2024}\\
\emph{Tolio - Compartir es Crecer}

\end{document}
